%% Main File
\documentclass{article}
\usepackage{style} % Required for inserting images

\chead{Zweck}
\lhead{Name}
\rhead{\today}
\cfoot{\thepage\ / \pageref{LastPage}}

\begin{document}
\section{Aufgabe N}
\lipsum[1]
\newline
\texttt{tcolorbox}:
\begin{tcolorbox}[colback=red!5!white,colframe=red!60!black,title=Überschrift]
    Inhalt
\end{tcolorbox}
\texttt{hyperref}: \href{https://overleaf.com/}{Overleaf Url}
\newline
$a^2 + b^2 = c^2$
$$a^2 + b^2 = c^2$$
\\
\texttt{amsfonts}:
\begin{itemize}
    \item \texttt{mathbb} $\mathbb{N}$ (natürliche Zahlen), $\mathbb{Z}$ (ganze Zahlen), $\mathbb{R}$ (reelle Zahlen) und $\mathbb{C}$ (komplexe Zahlen)
    \item \texttt{mathfrak} $\mathfrak{g}$ eine Frakturkleinbuchstabe "g" dar.
\end{itemize}
\texttt{amssymb}:
\begin{itemize}
    \item $\in$ für "Element von", $\forall$ für "für alle" und $\exists$ für "es gibt".
    \item  $\emptyset$ für die leere Menge, $\subset$ für "ist eine Teilmenge von" und $\cup$ und $\cap$ für Vereinigung bzw. Schnittmenge.
    \item  $\neg$ für "nicht", $\wedge$ für "und" und $\vee$ für "oder".
    \item $\rightarrow$ für "impliziert" und $\leftrightarrow$ für "wenn und nur wenn".
\end{itemize}
\texttt{amsthm}:
\begin{theorem}
Theorem.
\end{theorem}
\begin{lemma}
Lemma
\end{lemma}
\begin{definition}
Definition
\end{definition}
\begin{example}
Beispiel
\end{example}
\section{}
\begin{theorem}
Theorem.
\end{theorem}
\begin{lemma}
Lemma
\end{lemma}
\begin{definition}
Definition
\end{definition}
\begin{example}
Beispiel
\end{example}

\lipsum[1]
\texttt{algorithm \& algpseudocodex}:
\begin{algorithm}
\caption{Berechnung der Summe der ersten $n$ natürlichen Zahlen}
\begin{algorithmic}[1]
\Procedure{SummeDerErstenN}{n}
    \State $sum \gets 0$
    \For{$i \gets 1$ \textbf{to} $n$}
        \State $sum \gets sum + i$
    \EndFor
    \State \textbf{return} $sum$
\EndProcedure
\end{algorithmic}
\end{algorithm}
\begin{algorithm}
\begin{algorithmic}[1]
\State $n \gets 5$
\State $ergebnis \gets$ \Call{SummeDerErstenN}{$n$}
\State \textbf{Ausgabe} "Die Summe der ersten $n$ natürlichen Zahlen ist $ergebnis$."
\end{algorithmic}
\end{algorithm}
\end{document}